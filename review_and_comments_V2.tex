\documentclass[11pt]{article}
%	options include 12pt or 11pt or 10pt
%	classes include article, report, book, letter, thesis

\usepackage{url}
\usepackage{fullpage}

\title{Explanations of Reviewers comments}
\date{\today}

\begin{document}
\maketitle
\section*{Reviewer: 1}


\textbf{Comments:}

\begin{enumerate}

\item\textbf{Recommendation: Review Again After Resubmission (Paper is not acceptable in its current form, but has merit. A major rewrite is required. Author should be encouraged to resubmit a rewritten version after the changes suggested in the Comments section have been completed.)}\\
We have revised the paper after taking into account all comments from two reviewers and the editor. For details, please see below our comments. For ease of reference, each of our response has been added after the corresponding comment from the reviewer(s).

\textbf{Comments:
The title of the paper is still not justified. Author need to improve this. The whole paper is based on fall detection as well as posture monitoring. This paper is not based on activities tracking or detection. Activities detection includes various kinds of activities such as sleeping, eating and toilet.}\\
 We have changed the title from: `Semi-Automatic Extraction of Training Examples from Sensor Readings for Tracking Human  Activities' to `Semi-Automatic Extraction of Training Examples from Sensor Readings for Fall Detection and Posture Monitoring'

\textbf{Additional Questions:}
\item\textbf{1. Is the topic appropriate for publication in these transactions?: Yes}\\
Thank  you

\item\textbf{2. Is the topic important to colleagues working in the field?: Yes}\\
Thank you
\item\textbf{1. Is the paper technically sound? If no, why not?: Yes}\\
Thank you

\item \textbf{2. Is the coverage of the topic sufficiently comprehensive and balanced? : Yes
}\\
Thank you

\item\textbf{3. How would you describe the technical depth of the paper?: Suitable only for an expert}\\
Thank you

\item\textbf{4. How would you rate the technical novelty of the paper?: Somewhat novel}\\
Thank you

\item \textbf{1. How would you rate the overall organization of the paper?: Could be improved
}\\
As shown in the highlighted version, we have revised the title, the abstract, introduction and related work sections. Also, we have added five references. One of the references is a very recent and comprehensive review on human activity monitoring using wearable sensors.
\item \textbf{2. Are the title and abstract satisfactory? : Yes
}\\
Thank you

\item\textbf{3. Is the length of the paper appropriate? If not, recommend how the length of the paper should be amended, including a possible target length for the final manuscript. : Yes}\\
Thank you

\item\textbf{4. Are symbols, terms, and concepts adequately defined?: Yes}\\
Thank you

\item\textbf{5. How do you rate the English usage?: Satisfactory}\\
Thank you

\item\textbf{6. Rate the Bibliography?: Unsatisfactory}\\
After a new literature search, we have added five references that are most relevant to our work. One of the references is a very recent and comprehensive review on human activity monitoring using wearable sensors. Please see highlighted version of the paper for details.

\item\textbf{1. How would you rate the technical contents of the paper? : good}\\
Thank you

\item \textbf{2. How would you rate the novelty of the paper? : slightly novel}\\
Thank you

\item\textbf{3. How would you rate the "literary" presentation of the paper?: partially accessible
}\\
Thank you

\item\textbf{4. How would you rate the appropriateness of this paper for publication in this IEEE Transactions? : good match}\\
Thank you

\item\textbf{Would you recommend this paper for a Best Paper Award?: No}\\
Thank you

\end{enumerate}

\section*{Reviewer: 2}
\begin{enumerate}
	\item \textbf{Recommendation: Publish in Minor, Required Changes (as noted in the Comments section. This rating may not be assigned for Sensors Letters.)
}\\
We have tried our best to make all recommended changes, but we are open for changes if the reviewer(s) could provide further suggestions.

\textbf{Comments:language has to be improved and work on the presentation style.
} \\
We have improved the language significantly; Especially, in the Sections I and II. The highlighted version shows the extent of changes.

\textbf{\textbf{Additional Questions:}}
\item\textbf{1. Is the topic appropriate for publication in these transactions?: Yes
}\\
Thank you

\item\textbf{2. Is the topic important to colleagues working in the field?: Yes}\\
Thank you

\item\textbf{Is the paper technically sound? if no, why not?: Yes}\\
Thank you
%
%\item\textbf{1. Is the paper technically sound? If no, why not?: Yes}
%Thank you

\item\textbf{2. Is the coverage of the topic sufficiently comprehensive and balanced? : Yes
}\\
Thank you

\item\textbf{3. How would you describe the technical depth of the paper?: Appropriate for the generally knowledgeable individual Working in the Field or a Related Field}
Thank you

\item\textbf{4. How would you rate the technical novelty of the paper?: Somewhat novel}\\
Thank you

\item\textbf{1. How would you rate the overall organization of the paper?: Satisfactory}\\
Thank you

\item\textbf{2. Are the title and abstract satisfactory? : Yes}\\
Thank you

\item\textbf{3. Is the length of the paper appropriate? If not, recommend how the length of the paper should be amended, including a possible target length for the final manuscript. :}\\
Thank you


\item\textbf{4. Are symbols, terms, and concepts adequately defined?: Yes
}\\
Thank you

\item\textbf{5. How do you rate the English usage?: Needs improvement}\\
We have improved the language significantly; Especially, in the Sections I and II. The highlighted version shows the extent of changes.

\item\textbf{6. Rate the Bibliography?: Satisfactory
}\\
Thank you
\item\textbf{1. How would you rate the technical contents of the paper? : fair}\\
Thank you
\item\textbf{2. How would you rate the novelty of the paper? : slightly novel}\\
Thank you

\item\textbf{3. How would you rate the "literary" presentation of the paper?: mostly accessible
}\\
Thank you
\item\textbf{4. How would you rate the appropriateness of this paper for publication in this IEEE Transactions? : good match}\\
Thank you

\item\textbf{Would you recommend this paper for a Best Paper Award?: No
}\\
Thank you

\end{enumerate}

%\section*{Reviewer: 1}
%
%
%\textbf{Comments:}
%
%\begin{enumerate}
%
%\item \textbf{Paper need to be improve in terms of technical annotations.}
%
%We have improved the terms and technical annotations to the best of our abilities. The 
%definitions and the abbreviations are corrected and used appropriately.  A native 
%English 
%speaker has edited the manuscript and corrected  grammatical mistakes and typos.  
%If there 
%were to be further improvements recommended by the reviewers, we are open to correct 
%them. 
%
%\item \textbf{Page 4 table 1 page 52 and 53. Twice the author said Fall left.}
%
%Fixed.
%
%\item \textbf{Fig 3 and 4 do not explain about the significance of different colors used 
%in 
%graphs.}
%
%A new legend and an explanation is provided for all traces in the graphs. 
%
%\item \textbf{Fig.8 Visualization is quite poor.}
%
%We have re-complied the Figure 8 with higher resolution. It is now Figure 7 in the 
%revised manuscript.   
%
%\item \textbf{The data used in the paper for analysis, is it realistic system data? What 
%was the testing environment?}
%
%The data used in the manuscript is collected from the system described in Section III. 
%The 
%raw data, pre-processed data, and code is available for public from 
%\url{http://muenergia.saminda.org}. The information about the datasets are available in 
%Section III. 
%
%\item \textbf{The analysis author have performed is it real time analysis or offline 
%analysis?}
%
%The parameter learning of the proposed system is conducted offline. Once the training 
%examples are extracted from our proposed method, we have used the x-fold cross validation 
%to report the results. The fold number was chosen appropriately for the given dataset. 
%The offline training algorithms and configurations are provided in Section V. Once the 
%parameters are learned, we have fixed them, they are used in online setting to identify 
%and predict the event. We have used the same setup as given in Sections III and IV. We 
%have reported the results only from the x-fold cross validation.  
%
%\end{enumerate}
%
%
%\textbf{Additional Questions:}
%
%\begin{enumerate}
%
%\item \textbf{1. Is the topic appropriate for publication in these transactions?: Yes}
%
%Not applicable. 
%
%\item \textbf{2. Is the topic important to colleagues working in the field?: Yes}
%
%Not applicable.
%
%\item \textbf{1. Is the paper technically sound? If no, why not?: Yes}
%
%Not applicable.
%
%\item \textbf{2. Is the coverage of the topic sufficiently comprehensive and balanced? : 
%Important information is missing or superficially treated.}
%
%Due to the space limitation, some of the information, such as gradient calculation for 
%different classifiers, so forth, has been refereed to the original publications. We are 
%open to include more information, if the reviewers may require them. 
%
%\item \textbf{3. How would you describe the technical depth of the paper?: Suitable only 
%for an expert}
%
%Not applicable.
%
%\item \textbf{4. How would you rate the technical novelty of the paper?: Somewhat novel}
%
%Not applicable.
%
%\item \textbf{1. How would you rate the overall organization of the paper?: Satisfactory}
%
%Not applicable.
%
%\item \textbf{2. Are the title and abstract satisfactory? : No (explain) \\
%title and abstract explanation: Paper is more toward the tracking but abstract and title 
%claims the activity prediction and monitoring.}
%
%We have changed the abstract and the content of the paper to match the title. Our main 
%goal of the paper is the prediction of the activity. We are not considered the prediction 
%as tracking as our methods are based on classification.  
%
%\item \textbf{3. Is the length of the paper appropriate? If not, recommend how the length 
%of the paper should be amended, including a possible target length for the final 
%manuscript. : Yes}
%
%\end{enumerate}
%
%
%\textbf{Length of the paper recommendation:}
%
%\begin{enumerate}
%\setcounter{enumi}{9}
%\item \textbf{4. Are symbols, terms, and concepts adequately defined?: Not always}
%
%We have improved and unified the symbols, terms, and definitions to the best of our 
%abilities, and have 
%included 
%them appropriately.  If there 
%were to be further improvements recommended by the reviewers, we are open to correct 
%them. 
%
%
%
%\item \textbf{5. How do you rate the English usage?: Satisfactory}
%
%Not applicable.
%
%\item \textbf{6. Rate the Bibliography?: Satisfactory}
%
%Not applicable.
%
%\item \textbf{1. How would you rate the technical contents of the paper? : fair}
%
%Not applicable.
%
%\item \textbf{2. How would you rate the novelty of the paper? : slightly novel}
%
%Not applicable.
%
%\item \textbf{3. How would you rate the "literary" presentation of the paper?: partially 
%accessible}
%
%Not applicable.
%
%\item \textbf{4. How would you rate the appropriateness of this paper for publication in 
%this IEEE Transactions? : good match}
%
%Not applicable.
%
%\item \textbf{Would you recommend this paper for a Best Paper Award?: No}
%
%Not applicable.
%
%\end{enumerate}
%
%
%\section*{Reviewer: 2}
%
%\textbf{Comments:}
%
%\begin{enumerate}
%
%\item \textbf{Justify the design and advantages of the same. Grammar and presentation 
%format has to be modified.}
%
%We have empirically shown the design advantages in Section VII. The justification for the 
%layered architecture is conjectured for our datasets via experimentation. The experiments 
%have clearly shown that layered architecture outperform the baseline setups we have 
%considered. 
%
%As mentioned in the review 1, A native 
%English 
%speaker has edited the manuscript and corrected grammatical mistakes and typos.  
%If there 
%were to be further improvements recommended by the reviewers, we are open to correct 
%them.  
%
%\item \textbf{Comment on the novelty as well.}
%
%The novelty of our paper is the semi-automatic training example extraction procedures and 
%the layered classification network. As 
%mentioned in the related work section, all the methods have used a manual training data 
%annotation processes. These methods are cumbersome and consume a lot of time. Our 
%proposed 
%method significantly expedite this process. Only user involvement is at the end of the 
%process to interpret the cluster centroid, i.e. to mark it whether it belongs to a fall 
%or 
%non-fall region. At this state the user involvement is unavoidable. When we initialize 
%the clustering with different seeds, the interpretation of the centers differs. For 
%example, if we were to run the clustering for 
%arbitrary initialization, the first cluster may points to all fall events data points. If 
%we were to run it for the second time, because of the arbitrary initialization of the 
%cluster centers, the first cluster now may point to all non-fall data points. Therefore, 
%we need a user at the end of the process to interpret the cluster centroids, hence, 
%semi-automatic.
%
%We have conjectured via experimentation that the proposed layered architecture 
%outperforms the baseline methods. As mentioned in Section VII, our architecture has 
%exploited the structure of the dataset. We hypothesis that this form of methods will be 
%suitable for similar problem. 
%
%\item \textbf{Please give more explanations of the title especially 'Semi-Automatic 
%Extraction... '}
%
%We have explained the semi-automatic training example extraction and its  benefits in 
%Section IV. 
%
%\end{enumerate}
%
%
%\textbf{Additional Questions:}
%
%\begin{enumerate}
%
%\item \textbf{1. Is the topic appropriate for publication in these transactions?: Yes}
%
%Not applicable.
%
%\item \textbf{2. Is the topic important to colleagues working in the field?: Yes}
%
%Not applicable.
%
%\item \textbf{1. Is the paper technically sound? If no, why not?: Yes}
%
%Not applicable.
%
%\item \textbf{2. Is the coverage of the topic sufficiently comprehensive and balanced? : 
%Treatment somewhat unbalanced, but not seriously so.}
%
%We have re-written some parts of the manuscript to overcome explanation deficiencies to 
%the best of our abilities. If there 
%were to be further improvements recommended by the reviewers, we are open to correct 
%them.
%
%\item \textbf{3. How would you describe the technical depth of the paper?: Appropriate 
%for the generally knowledgeable individual Working in the Field or a Related Field}
%
%Not applicable.
%
%\item \textbf{4. How would you rate the technical novelty of the paper?: Somewhat novel}
%
%Not applicable.
%
%\item \textbf{1. How would you rate the overall organization of the paper?: Could be 
%improved}
%
%We have corrected and organized the paper to the best of our abilities.  A native 
%English 
%speaker has edited the manuscript and corrected grammatical mistakes and typos.  
%If there 
%were to be further improvements recommended by the reviewers, we are open to correct 
%them. 
 %
%
%\item \textbf{2. Are the title and abstract satisfactory? : Yes}
%
%Not applicable.
%
%\end{enumerate}
%
%\textbf{Title and abstract explanation:}
%
%
%\begin{enumerate}
%\setcounter{enumi}{8}
%\item \textbf{3. Is the length of the paper appropriate? If not, recommend how the length 
%of the paper should be amended, including a possible target length for the final 
%manuscript. : Yes}
%
%Not applicable.
%
%\end{enumerate}
%
%
%\textbf{Length of the paper recommendation:}
%
%\begin{enumerate}
%\setcounter{enumi}{9}
%
%\item  \textbf{4. Are symbols, terms, and concepts adequately defined?: Not always}
%
%As mentioned in review 1, we have improved and unified the symbols, terms, and 
%definitions to the best of our 
%abilities, and have 
%included 
%them appropriately.  If there 
%were to be further improvements recommended by the reviewers, we are open to correct 
%them. 
%
%\item \textbf{5. How do you rate the English usage?: Needs improvement}
%
%A native 
%English 
%speaker has edited the manuscript and corrected  grammatical mistakes and typos.  
%If there 
%were to be further improvements recommended by the reviewers, we are open to correct and 
%include  
%them. 
%
%\item \textbf{6. Rate the Bibliography?: Satisfactory}
%
%Not applicable.
%
%\item \textbf{1. How would you rate the technical contents of the paper? : fair}
%
%Not applicable.
%
%\item \textbf{2. How would you rate the novelty of the paper? : sufficiently novel}
%
%Not applicable.
%
%\item \textbf{3. How would you rate the "literary" presentation of the paper?: partially 
%accessible}
%
%Not applicable.
%
%\item  \textbf{4. How would you rate the appropriateness of this paper for publication in 
%this IEEE Transactions? : good match}
%
%Not applicable.
%
%\item \textbf{Would you recommend this paper for a Best Paper Award?: No}
%
%Not applicable.
%
%\end{enumerate}


\end{document}