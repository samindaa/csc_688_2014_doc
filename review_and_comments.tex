\documentclass[12pt]{article}
%	options include 12pt or 11pt or 10pt
%	classes include article, report, book, letter, thesis

\usepackage{url}

\title{Explanations of Reviewers comments}
\date{\today}

\begin{document}
\maketitle

\section*{Reviewer: 1}


\textbf{Comments:}

\begin{enumerate}

\item \textbf{Paper need to be improve in terms of technical annotations.}

We have improved the terms and technical annotations to the best of our abilities. The 
definitions and the abbreviations are corrected and used appropriately.  A native 
English 
speaker has edited the manuscript and corrected most of grammatical mistakes and typos.  
If there 
were to be further improvements recommended by the reviewers, we are open to correct 
them. 

\item \textbf{Page 4 table 1 page 52 and 53. Twice the author said Fall left.}

Fixed.

\item \textbf{Fig 3 and 4 do not explain about the significance of different colors used 
in 
graphs.}

A new legend and an explanation is provided for all traces in the graphs. 

\item \textbf{Fig.8 Visualization is quite poor.}

We have re-complied the Figure 8 with higher resolution. It is now Figure 7 in the 
revised manuscript.   

\item \textbf{The data used in the paper for analysis, is it realistic system data? What 
was the testing environment?}

The data used in the manuscript is collected from the system described in Section III. 
The 
raw data, pre-processed data, and code is available for public from 
\url{http://muenergia.saminda.org}. The information about the datasets are available in 
Section III. 

\item \textbf{The analysis author have performed is it real time analysis or offline 
analysis?}

The parameter learning of the proposed system is conducted offline. Once the training 
examples are extracted from our proposed method, we have used the x-fold cross validation 
to report the results. The fold number was chosen appropriately for the given dataset. 
The offline training algorithms and configurations are provided in Section V. Once the 
parameters are learned, we have fixed them, they are used in online setting to identify 
and predict the event. We have used the same setup as given in Sections III and IV. We 
have reported the results only from the x-fold cross validation.  

\end{enumerate}


\textbf{Additional Questions:}

\begin{enumerate}

\item \textbf{1. Is the topic appropriate for publication in these transactions?: Yes}

\item \textbf{2. Is the topic important to colleagues working in the field?: Yes}

\item \textbf{1. Is the paper technically sound? If no, why not?: Yes}


\item \textbf{2. Is the coverage of the topic sufficiently comprehensive and balanced? : 
Important information is missing or superficially treated.}

\item \textbf{3. How would you describe the technical depth of the paper?: Suitable only 
for an expert}

\item \textbf{4. How would you rate the technical novelty of the paper?: Somewhat novel}

\item \textbf{1. How would you rate the overall organization of the paper?: Satisfactory}

\item \textbf{2. Are the title and abstract satisfactory? : No (explain) \\
title and abstract explanation: Paper is more toward the tracking but abstract and title claims the activity prediction and monitoring.}

\item \textbf{3. Is the length of the paper appropriate? If not, recommend how the length 
of the paper should be amended, including a possible target length for the final 
manuscript. : Yes}

\textbf{length of the paper recommendation:}

\item \textbf{4. Are symbols, terms, and concepts adequately defined?: Not always}

\item \textbf{5. How do you rate the English usage?: Satisfactory}

\item \textbf{6. Rate the Bibliography?: Satisfactory}

\item \textbf{1. How would you rate the technical contents of the paper? : fair}

\item \textbf{2. How would you rate the novelty of the paper? : slightly novel}

\item \textbf{3. How would you rate the "literary" presentation of the paper?: partially 
accessible}

\item \textbf{4. How would you rate the appropriateness of this paper for publication in 
this IEEE Transactions? : good match}

\item \textbf{Would you recommend this paper for a Best Paper Award?: No}

\end{enumerate}


\section*{Reviewer: 2}

Comments:
\textbf{Justify the design and advantages of the same. Grammar and presentation format has to be modified.}
\textbf{Comment on the novelty as well.}
\textbf{Please give more explanations of the title especially 'Semi-Automatic Extraction... '}

Additional Questions:

\textbf{1. Is the topic appropriate for publication in these transactions?: Yes}

\textbf{2. Is the topic important to colleagues working in the field?: Yes}

\textbf{1. Is the paper technically sound? If no, why not?: Yes}


\textbf{2. Is the coverage of the topic sufficiently comprehensive and balanced? : Treatment somewhat unbalanced, but not seriously so.}

\textbf{3. How would you describe the technical depth of the paper?: Appropriate for the generally knowledgeable individual Working in the Field or a Related Field}

\textbf{4. How would you rate the technical novelty of the paper?: Somewhat novel}

\textbf{1. How would you rate the overall organization of the paper?: Could be improved}

\textbf{2. Are the title and abstract satisfactory? : Yes}

title and abstract explanation:

\textbf{3. Is the length of the paper appropriate? If not, recommend how the length of the paper should be amended, including a possible target length for the final manuscript. : Yes}

length of the paper recommendation:

\textbf{4. Are symbols, terms, and concepts adequately defined?: Not always}

\textbf{5. How do you rate the English usage?: Needs improvement}

\textbf{6. Rate the Bibliography?: Satisfactory}

\textbf{1. How would you rate the technical contents of the paper? : fair}

\textbf{2. How would you rate the novelty of the paper? : sufficiently novel}

\textbf{3. How would you rate the "literary" presentation of the paper?: partially accessible}

\textbf{4. How would you rate the appropriateness of this paper for publication in this IEEE Transactions? : good match}

\textbf{Would you recommend this paper for a Best Paper Award?: No}

\end{document}